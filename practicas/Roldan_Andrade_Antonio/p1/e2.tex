
\begin{lstlisting}
    /*
    FUNCION DE FACTIBILIDAD
     Diseñe una funcion de factibilidad explicita y descrıbala a continuacion.
     Entiendo que la funcion sirve para comprobar si es posible que se coloque algo en una celda
     (una defensa en x lugar)
     Parametros: 
        Row-> entero que representa una fila en la matriz del mapa
        col-> entero que representa una col en la matriz del mapa
        obstacles-> Recibe la lista de defensas colocadas en el mapa para iterar y comprobar que no coincidan
        Defenses-> Recibe la lista de obstacles en el mapa para iterar y comprobar que no coincidan
        freeCells -> Matriz con el numero de celdas libres
        mapHeight -> altura del mapa (eje z)
        mapWidth -> anchura del mapa (eje x)
        int nCellsWidth-> numero de celdas en total del eje x
        int nCellsHeight-> numero de celdas en total del eje z
        Defense d ==> Recibe una referencia a la defensa a colocar en la celda
        PREGUNTAR QUE SIGNIFICA QUE SEA UNA FUNCION DE FACTIBILIDAD EXPLICITA
    */
    bool funcion_factibilidad(int row, int col, List<Object*> obstacles, List<Defense*> defenses,
    bool **freeCells,float mapHeight, float mapWidth,int nCellsWidth, int nCellsHeight, const Defense& d){
        //definicion de variables
        float cellWidth = mapWidth / nCellsWidth; //anchura de la celula
        float cellHeight = mapHeight / nCellsHeight;//altura de la celula
        Vector3 p = cellCenterToPosition(row,col,cellWidth,cellHeight); //Creamos la posicion con los datos dados
        bool token = true;
    
        //Primeiro comprobamos que la celda no esta en ninguna posicion limite
        if(p.x + d.radio > mapWidth ||p.x - d.radio < 0 ||
            p.y + d.radio > mapHeight ||p.y - d.radio < 0)
        {
            return false; //si se cumple alguna condicion la defensa no cabe al alcanzar posiciones limite del mapa
        }    
        else{//Si no cumple el primer requisito no sera necesario continuar, en cambio , si lo cumple entonces:
            //Comprobaremos que no colisionan con las defensas/obstaculos que ya estan colocadas
            
            
            for(auto i : obstacles){
                //Colisionara en caso de que las distancias entre puntos centrales de los obstaculos
                //sea menor que los radios de la defensa a colocar y el obstaculo
                if((d.radio + i->radio)>_distance(p,i->position))
                    token = false;
            }
            if(token){//si ya se ha detectado que no es posible colocarla en un obstaculo
                      //para que comprobar las defensas
                //Se comprobara de forma similar a los obstaculos con las defensas
                for (auto i: defenses){
                    if((d.radio + i->radio)>_distance(p,i->position))
                        token = false;
                }
            }
        }
        return token;      
    }
\end{lstlisting}