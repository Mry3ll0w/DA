\begin{lstlisting}
// Recibe la struct tsp (que tiene la lista de defensas a colocar etc) y el maximo numero de ases
//Se asume que la TSP ya se ha creado e inicializado con exito
std::pair<int,TSP> max_beneficio(TSP& tsp,const int& max_ases ){
    
    //Para poder trabajar con la matriz con comodidad usaremos un vector de pairs donde el primero sera el coste y el segundo la valoracion
    std::vector<std::pair<int,int>> v_coste_valor;

    for (auto i: tsp.def_val_list)//Lo subimos sin comprobar ordenacion, ya que esta viene ordenada
    {
        v_coste_valor.push_back(std::make_pair(i.d->cost,i.valoracion));    
    }
    


    //1) Inicializamos la fila inicial de la matriz_tsp

    //comenzamos la insercion de la fila de 0 capacidad
    for(int j = 0; j < max_ases+1; ++j){
        
        if (j < tsp.def_val_list.begin()->d->cost) //Comprobamos que la primera defensa se pueda pagar con 0 ases
        {
            tsp.matriz_tsp[0][j]=0;
        }
        
        else{ //En caso de poder pagarla guardamos la valoracion de la misma
            tsp.matriz_tsp[0][j]=tsp.def_val_list.begin()->valoracion; // metemos el valor del elemento que cabe en 0 
        }
    }
    
    //2) Rellenamos el resto de filas
    
    //comenzaremos en el segundo elemento ya que el primero la matriz ya esta relleno
    for ( int i=1; i < tsp.def_val_list.size() ; i++ ){

        for ( int j=0; j<max_ases+1; j++){
          
            if (j <  v_coste_valor[i].first)
            {
                tsp.matriz_tsp[i][j]=tsp.matriz_tsp[i-1][j]; //si el coste es mayor que el nº de ases, metemos el anterior  
            }
            else{
                
                tsp.matriz_tsp[i][j]=std::max(tsp.matriz_tsp[i-1][j],tsp.matriz_tsp[i-1][j-v_coste_valor[i].first]+v_coste_valor[i].second);
            
            }

        }
    
    }

    return std::make_pair(tsp.matriz_tsp[tsp.def_val_list.size()][max_ases],tsp);//devolvemos el ultimo elemento de la matriz que contiene el max valor

}

\end{lstlisting}