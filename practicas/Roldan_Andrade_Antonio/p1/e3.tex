\begin{lstlisting}

struct celda_valoracion
{
    int row, col;
    float value;
    celda_valoracion(int r = 0, int c = 0, float v = 0) : row(r), col(c), value(v) {}
    
    //Overload ctor de copia
    celda_valoracion operator =(const celda_valoracion& c){
        this->row = c.row; this->col = c.col;
        return *this;
    }

    //Overload operator < (En caso de ordencacion por lista)
    bool operator <(const celda_valoracion& c){
        return value < c.value;
    }

};

void DEF_LIB_EXPORTED placeDefenses(bool **freeCells, int nCellsWidth, int nCellsHeight, float mapWidth, 
float mapHeight, std::list<Object *> obstacles, std::list<Defense *> defenses)
{

    float cellWidth = mapWidth / nCellsWidth;
    float cellHeight = mapHeight / nCellsHeight;

    int maxAttemps = 1000;
    std::vector<celda_valoracion> celdas_valoradas;
    int i, j, k;
    float aux;
    celda_valoracion auxCell;

    /* -------------------------------------------------------------------------- */
    /*                 ASINGACION DEL CENTRO DE EXTRACCION (DEF 0)                */
    /* -------------------------------------------------------------------------- */

    //1) Obtenemos que celdas estan libres tras la colocacion de los obstaculos
    for (int i = 0; i < nCellsHeight; i++){
        
        for (int j = 0; j < nCellsWidth; j++)
        {
            if (freeCells[i][j] != false){
                
                celdas_valoradas.push_back(celda_valoracion(i, j, cellValueExtractionCenter(i, j, freeCells, nCellsWidth, nCellsHeight, 
                mapWidth, mapHeight, obstacles)));

            }
                
                
        }

    }
    
    //Ordenacion de las celdas valoradas aplicando los constructores de vector y lista 
    
    std::list<celda_valoracion>aux_lista(celdas_valoradas.begin(),celdas_valoradas.end());//Creo una lista con los elementos de celdas
    aux_lista.sort();//Ordeno los elementos O(n) = n · log n
    celdas_valoradas = std::vector<celda_valoracion>(aux_lista.begin(),aux_lista.end());//los copio a la lista


    bool placed = false;
    celda_valoracion solution;
    List<Defense *>::iterator currentDefense = defenses.begin();

    //Algoritmo devorador para centro de extraccion
    while (!placed && !celdas_valoradas.empty())
    {
        solution = celdas_valoradas.back();
        celdas_valoradas.pop_back();
        if (funcion_factibilidad(defenses, *(*currentDefense), obstacles, mapHeight,
                     cellWidth, cellHeight, mapWidth, solution.row, solution.col, freeCells))
        {
            placed = true;
            freeCells[solution.row][solution.col] = false;
            (*currentDefense)->position = cellCenterToPosition(solution.row, solution.col, cellWidth, cellHeight);
        }
        
    }
    //Copia de los candidatos

    std::vector<celda_valoracion> celdas_libres_aux;

    //Insertamos las celdas aun libres en el vector 
    for (i = 0; i < nCellsHeight; i++)
        for (j = 0; j < nCellsWidth; j++)
        {
            if (freeCells[i][j])
                celdas_libres_aux.push_back(celda_valoracion(i, j, defaultCellValue(i,j,freeCells,nCellsWidth,nCellsHeight,mapWidth,mapHeight,obstacles,defenses)));
        }

    //Ordenamos los elementos de forma similar a lo hecho con anterioridad
    std::list<celda_valoracion>aux_lista2(celdas_libres_aux.begin(),celdas_libres_aux.end());//Creo una lista con los elementos de celdas
    aux_lista2.sort();//Ordeno los elementos O(n) = n · log n
    celdas_libres_aux = std::vector<celda_valoracion>(aux_lista2.begin(),aux_lista2.end());//los copio a la lista

    //Copiamos el contenido a las celdas originales para la insercion en el mapa de las defensas
    celdas_valoradas = celdas_libres_aux;

    std::vector<celda_valoracion>::iterator it;
    
    currentDefense++;//pasamos al segundo elemento ya que el primero ya se ha colocado (centro de extraccion)

    while (currentDefense != defenses.end())
    {
        
        placed = false;
        it = celdas_libres_aux.end();
        while (!placed && !celdas_valoradas.empty())
        {
            
            solution = celdas_valoradas.back();
            celdas_valoradas.pop_back();
            if (funcion_factibilidad(defenses, *(*currentDefense), obstacles, mapHeight, cellWidth, cellHeight, mapWidth, solution.row, solution.col, freeCells))
            {
                placed = true;
                (*currentDefense)->position = cellCenterToPosition(solution.row, solution.col, cellWidth, cellHeight);
                
                    it--;
                
                celdas_libres_aux.erase(it);
            }
        }
        celdas_valoradas.clear();
        celdas_valoradas = celdas_libres_aux;
        currentDefense++;
    }
}
\end{lstlisting}