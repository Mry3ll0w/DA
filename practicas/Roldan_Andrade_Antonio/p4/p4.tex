\documentclass[]{article}

\usepackage[left=2.00cm, right=2.00cm, top=2.00cm, bottom=2.00cm]{geometry}
\usepackage[spanish,es-noshorthands]{babel}
\usepackage[utf8]{inputenc} % para tildes y ñ

%opening
\title{Práctica 4. Exploración de grafos}
\author{Antonio Roldán Andrade \\ % mantenga las dos barras al final de la línea y este comentario
antonio.roldanandrade@alum.uca.es \\ % mantenga las dos barras al final de la línea y este comentario
Teléfono: 611404497 \\ % mantenga las dos barras al final de la linea y este comentario
NIF: 49562495W \\ % mantenga las dos barras al final de la línea y este comentario
}


\begin{document}

\maketitle

%\begin{abstract}
%\end{abstract}

% Ejemplo de ecuación a trozos
%
%$f(i,j)=\left\{ 
%  \begin{array}{lcr}
%      i + j & si & i < j \\ % caso 1
%      i + 7 & si & i = 1 \\ % caso 2
%      2 & si & i \geq j     % caso 3
%  \end{array}
%\right.$

\begin{enumerate}
\item Comente el funcionamiento del algoritmo y describa las estructuras necesarias para llevar a cabo su implementación.

Escriba aquí su respuesta al ejercicio 1.

valor = (def.damage*def.attacksPerSecond*def.health*def.range) / (def.dispersion);

\item Incluya a continuación el código fuente relevante del algoritmo.

Escriba aquí su respuesta al ejercicio 2.

Una matriz reprensentada con un vector de vectores de tipo float, de la siguiente forma,
std::vector<std::vector<float>>& TSP, cada vector representará uno los valores y otro los costes de las defensas,
respectivamente.

Una lista donde están almacenadas las defensas, que nos permite recorrerlas con mayor facilidad.
Nos permite también las inserciones y borrados con mayor rapidez.


\end{enumerate}

Todo el material incluido en esta memoria y en los ficheros asociados es de mi autoría o ha sido facilitado por los profesores de la asignatura. Haciendo entrega de esta práctica confirmo que he leído la normativa de la asignatura, incluido el punto que respecta al uso de material no original.

\end{document}
