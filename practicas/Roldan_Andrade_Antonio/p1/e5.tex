
\begin{lstlisting}
    
float defaultCellValue(int row, int col, bool** freeCells, int nCellsWidth, int nCellsHeight
    , float mapWidth, float mapHeight, List<Object*> obstacles, List<Defense*> defenses) {
    	
    float cellWidth = mapWidth / nCellsWidth;
    float cellHeight = mapHeight / nCellsHeight;

    Vector3 cellPosition((col * cellWidth) + cellWidth * 0.5f, (row * cellHeight) + cellHeight * 0.5f, 0);
    	
    float val = 0;
    for (List<Object*>::iterator it=obstacles.begin(); it != obstacles.end(); ++it) {
	    val += _distance(cellPosition, (*it)->position);
    }

    return val;
}

float cellValueExtractionCenter(int row, int col, bool **freeCells, int nCellsWidth, int nCellsHeight, 
float mapWidth, float mapHeight, List<Object *> obstacles)
{
    
    float cellWidth = mapWidth / nCellsWidth; //anchura de la celula
    float cellHeight = mapHeight / nCellsHeight;//altura de la celula
    //Aplicamos el criterio ==> cuanto mas cerca de un obstaculo mejor 
    //Usamos el tipo vector 3 para una comparativa (si esta vacia o no de forma mas sencilla)
    Vector3 t_posicion = cellCenterToPosition(row,col,cellWidth,cellHeight);
    float value = 0;
    for(auto i: obstacles){
        value+=i->position.subtract(t_posicion).length();
        //value+=_distance(t_position,i.position); 
        //me da fallos a pesar de que esto representa lo mismo que arriba
    }
    
    //el que tenga menor valor tendra mas obstaculos cerca ==> mayor puntuacion
    //por tanto lo invierto para tener mayor puntuacion
    return 1/value; // implemente aqui la funci�n que asigna valores a las celdas
}
\end{lstlisting}