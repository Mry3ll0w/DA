Escriba aquí su respuesta al ejercicio 1. 

Realizo una ponderación en caso de lo que considere de mayor valor o menor sobre un 100 por ciento, como
podemos observar en las celdas correspondientes a los centros de extracción lo valoraremos en un 1 por ciento

\begin{lstlisting}
float cellValueExtr(int row, int col, bool **freeCells, int nCellsWidth, int nCellsHeight, float mapWidth, float mapHeight, List<Object *> obstacles)
{
    float cellWidth = mapWidth / nCellsWidth;
    float cellHeight = mapHeight / nCellsHeight;
    int bestRow = nCellsHeight / 2, bestCol = nCellsWidth / 2;
    Vector3 posicion = cellCenterToPosition(row, col, cellWidth, cellHeight);
    Vector3 bestPos = cellCenterToPosition(bestRow, bestCol, cellWidth, cellHeight);
    float value;

    value = bestPos.length() - _distance(posicion, bestPos);

    if (row < 3 || row > nCellsHeight - 3 || col > nCellsWidth - 3 || col < 3)
        value = value * 0.01;

    return value;
}
\end{lstlisting}

\begin{figure}
\centering
\includegraphics[width=0.7\linewidth]{./defenseValueCellsHead} % no es necesario especificar la extensión del archivo que contiene la imagen
\caption{Estrategia devoradora para la mina}
\label{fig:defenseValueCellsHead}
\end{figure}