\documentclass[]{article}

\usepackage[left=2.00cm, right=2.00cm, top=2.00cm, bottom=2.00cm]{geometry}
\usepackage[spanish,es-noshorthands]{babel}
\usepackage[utf8]{inputenc} % para tildes y ñ
\usepackage{graphicx} % para las figuras
\usepackage{xcolor}
\usepackage{listings} % para el código fuente en c++

\lstdefinestyle{customc}{
  belowcaptionskip=1\baselineskip,
  breaklines=true,
  frame=single,
  xleftmargin=\parindent,
  language=C++,
  showstringspaces=false,
  basicstyle=\footnotesize\ttfamily,
  keywordstyle=\bfseries\color{green!40!black},
  commentstyle=\itshape\color{gray!40!gray},
  identifierstyle=\color{black},
  stringstyle=\color{orange},
}
\lstset{style=customc}


%opening
\title{Práctica 2. Programación dinámica}
\author{Antonio Roldán Andrade \\ % mantenga las dos barras al final de la línea y este comentario
antonio.roldanandrade@alum.uca.es \\ % mantenga las dos barras al final de la línea y este comentario
Teléfono: 611404497 \\ % mantenga las dos barras al final de la linea y este comentario
NIF: 49562495W \\ % mantenga las dos barras al final de la línea y este comentario
}


\begin{document}

\maketitle

%\begin{abstract}
%\end{abstract}

% Ejemplo de ecuación a trozos
%
%$f(i,j)=\left\{ 
%  \begin{array}{lcr}
%      i + j & si & i < j \\ % caso 1
%      i + 7 & si & i = 1 \\ % caso 2
%      2 & si & i \geq j     % caso 3
%  \end{array}
%\right.$

\begin{enumerate}
\item Formalice a continuación y describa la función que asigna un determinado valor a cada uno de los tipos de defensas.

Escriba aquí su respuesta al ejercicio 1.

valor = (def.damage*def.attacksPerSecond*def.health*def.range) / (def.dispersion);

\item Describa la estructura o estructuras necesarias para representar la tabla de subproblemas resueltos.

Escriba aquí su respuesta al ejercicio 2.

Una matriz reprensentada con un vector de vectores de tipo float, de la siguiente forma,
std::vector<std::vector<float>>& TSP, cada vector representará uno los valores y otro los costes de las defensas,
respectivamente.

Una lista donde están almacenadas las defensas, que nos permite recorrerlas con mayor facilidad.
Nos permite también las inserciones y borrados con mayor rapidez.

\item En base a los dos ejercicios anteriores, diseñe un algoritmo que determine el máximo beneficio posible a obtener dada una combinación de defensas y \emph{ases} disponibles. Muestre a continuación el código relevante.

Escriba aquí su respuesta al ejercicio 3.

\item Diseñe un algoritmo que recupere la combinación óptima de defensas a partir del contenido de la tabla de subproblemas resueltos. Muestre a continuación el código relevante.

\begin{lstlisting}
std::list<Defense*> recupera_defensas(const std::vector<std::vector<int> >& tsp,
const std::list<defensa_valoracion>& def_val,const int& filas, const int& cols,std::list<Defense*>defenses){

    //Almacenara las defensas que se han usado
    std::list<Defense*>sol;

    int i = filas - 2;
    int j = cols - (*defenses.begin())->cost; //Tenemos que eliminar el coste de la primera defense (centro de extraccion), que ademas se añade en cualquier caso

    List<Defense*>::iterator it = defenses.end();

    it--;   //Iniciamos en la posicion anterior a la ultima defensa (seria fin - 1)
    
    //Se Recorrera inversamente ya que partimos del beneficio maximo que se encuentra en 
    //la ultima posicion de la matriz
    while(i > 0)
    {
        if(tsp[i][j] != tsp[i-1][j])
        {
            j = j - (*it)->cost;
            sol.push_back(*(it));
        }
        i--;
        it--;
    }
    if(tsp[0][j] != 0) // En caso de que la primera posicion de la fila 0 sea != de 0 significa que la existe una defensa que tiene coste 0 y entrara en la lista
        sol.push_back((*it));//Insertamos dicha defensa

    
return sol; //Devolvemos la lista de defensas
}
\end{lstlisting}

\end{enumerate}

Todo el material incluido en esta memoria y en los ficheros asociados es de mi autoría o ha sido facilitado por los profesores de la asignatura. Haciendo entrega de este documento confirmo que he leído la normativa de la asignatura, incluido el punto que respecta al uso de material no original.

\end{document}
