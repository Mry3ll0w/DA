%Escriba aquí su respuesta al ejercicio 1. 
Para realizar este ejercicio he determinado usar el criterio de cercania con respecto a los 
obstaculos distribuidos en el mapa, es decir, cuanto mas cerca a los obstaculos
este la celda dada mejor puntuacion.
La puntuación se calcula en base a la suma de las distancias que separan a posicion creada a partir 
de un objeto tipo Vector3 y la funcion cellCenterToPosition, con esto podremos calcular la distancia
con respecto a la lista de obstaculos que se reciben como parametro.Por tanto la puntuación sera
el inverso de la suma de las distancias.
Anotación: El código de esta función esta "incrustado" en el ejercicio nº 5.

\begin{figure}
\centering
\includegraphics[width=0.7\linewidth]{./defenseValueCellsHead} 
 no es necesario especificar la extensión del archivo que contiene la imagen
\caption{Estrategia devoradora para la mina}
\label{fig:defenseValueCellsHead}
\end{figure}