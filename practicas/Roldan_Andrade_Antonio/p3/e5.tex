Analisis Temporal de los Algoritmos usados para la ordenacion de las defensas:

1)MERGESORT/Ordenacion Por Fusion

    a) Uso de Mergesort en la seleccion del centro de extraccion:
        Si consideramos que el algoritmo de ordenacion por fusion se aplica en una matriz de f filas y 
        c columnas cuadradadas, entonces tendremos que algoritmo se aplicara sobre n celdas.
        n = f x c (celdas)

        Sabemos que se aplica la ordenacion por fusion la cual es de orden O(n^2-n/2),
        este se aplica sobre las 2 mitades en las que se divide el vector de entrada, 
        por lo que se realizan n comparaciones en las 2 mitades.

        Por ultimo se fusionan los dos vectores, los cuales hacen n comparaciones.

        En conclusion:(Se define como optimo el umbral n0)

            |  si n <= n0  entonces se hacen n^2-n/2 llamadas a la operacion patron1 
            |
    t(n)=  |
           |
           |
            | si n > n0 entonces se hacen 2xt(n/2)+n llamadas a la operacion patron
            | 

        Llegando a la conclusion (tras resolver la recurrencia ) de que para la 
        ordenacion de las celdas para la seleccion del centro de extraccion es de
        O(n log(n))) en el peor caso.

    b) Ordenacion de las defensas
        La conclusion es la misma que la expuesta de forma anterior, solo que esta vez
        se aplicarian en vez de a n celdas a n defensas.

    Por tanto: t(n)= 2x n log n

2)QUICKSORT/Ordenacion Rapida

    El algoritmo de ordenacio rapida consta de 2 funciones.
    
    a.1)Funcion de particion: Ordena los elementos hasta el pivote n ,este se coloca 
    de forma "aleaotoria".
    Llamando a la ordenacion con el pivote en la mitad izquierda y despues en la mitad
    superior.

    a.2) Funcion de ordenacion quicksort: llama n veces al algoritmo hasta que el vector 
    este ordenado .

    En conclusion:

    
            |  si high >= low  entonces se hacen n^2-n/2 llamadas a la operacion patron1 
            |
    t(n)=  |
           |
           |
            | si low <= high entonces se hacen 2xt(n/2)+n llamadas a la operacion patron
            | 
    
    
        Llegando a la conclusion (tras resolver la recurrencia ) de que para la 
        ordenacion de las celdas para la seleccion del centro de extraccion, al igual
        que la ordenacion de las defensas es de O(n log(n))) en el peor caso.
        Por tanto: t(n)= 2x n log n