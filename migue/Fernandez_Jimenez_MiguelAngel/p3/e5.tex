Escriba aquí su respuesta al ejercicio 5.

1)Ordenación por fusión.

La idea detrás del divide y vencerás, es como el mismo nombre lo indica, dividir en sub-problemas que más adelante resuelvan el problema final. 
En el caso del Merge Sort, el lista será dividido en dos sub-listados, luego estos sub-listados serán nuevamente divididos
y así sucesivamente, hasta que llega a un listado de dos elementos que puede ser fácilmente ordenado. Luego de llegar a estos sub-listados pequeños de 1 o 2 elementos,
se comienzan a mezclar los sub-listados formando nuevos listados ordenados. Este proceso se repite hasta llegar al listados original que en última instancia, quedará ordenado.

Todo este proceso, tiene una complejidad O(nlog2n) dado que la altura del árbol que se forma al dividir en
sublistas el listado original es log2n donde n es la cantidad de elementos y en cada nivel del árbol 
hay que iterar sobre todos los elementos, la complejidad es O(n*log2n) 

 2)Ordenación rápida.

 Elegimos un elemento del conjunto de elementos a ordenar que llamaremos pivote.
 Situamos los demás elementos de la lista a cada lado del pivote, a un lado los menos que él y al otro los mayores, los que
 sean iguales se pueden colocar en cualquiera de los lados, ocupando así el pivote su lugar correspondiente.
 La lista se queda separada en dos sublistas, este proceso se repite de forma recursiva, la eficiencia depende de la posición en la que
 se termine el pivote elegido.

 Mejor caso -> pivote termina en el centro de la lista, O(n*logn).
 Peor caso -> pivote termina en un extremo de la lista, O(n^2).
 Caso promedio -> O(n*logn).