Para hacer las pruebas de caja negra probaremos con:
1) std::vector<int> v de 2 elementos
2) Un std::vector<int> v de longitud cambiante de entre 10 y 1000 elementos 

A continuación adjunto el codigo usado en las pruebas del mismo:

\begin{lstlisting}

//Se corresponde a la prueba con el valor limite de 2 elementos a ordenar
//Postcondicion==> Devuelve un booleano true si el vector se encuentra ordenado, en caso contrario sera falso

bool prueba_vec1(){
    
    bool ordenado = true; 
    std::vector<int>v={9,7};
    std::vector<int>v2={7,9};//Contiene el vector final ordenado

    
    //mergeSort(v,0,v.size()-1);
    quickSort(v,0,v.size()-1);
    // ---------- descomentar/comentar el algoritmo que se desee testear ---------- 
    
    for(size_t i=0;i<v.size();i++){
        if(v[i]!= v2[i]){
            ordenado= false;
        }
    }
    
    return ordenado;

}


//Se corresponde a la comprobacion con todos los arrays dinamicos con los elementos del 1 a 1000
//Postcondicion==> Devuelve un booleano true si el vector se encuentra ordenado, en caso contrario sera falso

bool prueba_vec_dinamico(){
    std::vector<int>v,v_sorted;
    bool ordered = true;
    size_t tam = 10;//Empezamos con vectores de tamaño

    while (tam < 1000 && ordered)
    {   
        //1)inicializamos el vector con los n primeros elementos
        v= std::vector<int>(tam);
        for (size_t i = 0; i < tam; i++)
        {
            v[i]=i+1;
        }
        v_sorted = v; // En ese momento v ya esta ordenado [1...tam]
        
        //2)Hacemos un shuffle de v
        std::shuffle(v.begin(), v.end(), std::random_device());
        
        //3)Llamaremos al sort que queramos y comprobamos  si falla o no
        //quickSort(v,0,v.size()-1);
        mergeSort(v,0,v.size()-1);
        // ---------- descomentar/comentar el algoritmo que se desee testear ---------- 
    

        for(size_t i = 0; i < tam && ordered; i++){
            if(v[i]!=v_sorted[i]){
                ordered = false;
            }
        }


        //Incrementamos tam para el siguiente vector
        tam++;
    }
    
return ordered;
}
\end{lstlisting}