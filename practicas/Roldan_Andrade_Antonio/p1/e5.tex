La defincion de esta funcion se halla en el ejercicio 1, ya que ambos usan la misma.
El codigo en cuestion: 
\begin{lstlisting}
    float cellValue(int row, int col, bool** freeCells, int nCellsWidth, int nCellsHeight
	, float mapWidth, float mapHeight, List<Object*> obstacles, List<Defense*> defenses) {
	//De forma similar al ejercicio inferior
    float cellWidth = mapWidth / nCellsWidth; //anchura de la celula
    float cellHeight = mapHeight / nCellsHeight;//altura de la celula
    /*Aplicamos el criterio ==> cuanto mas cerca de un obstaculo mejor 
    */
    //Usamos el tipo vector 3 para una comparativa (si esta vacia o no de forma mas sencilla)
    Vector3 t_posicion = cellCenterToPosition(row,col,cellWidth,cellHeight);
    float value = 0;
    for(auto i: obstacles){
        value+=i->position.subtract(t_posicion).length();
        //value+=_distance(t_position,i.position); 
        //me da fallos a pesar de que esto representa lo mismo que arriba
    }
    //el que tenga menor valor tendra mas obstaculos cerca ==> mayor puntuacion
    //por tanto lo invierto para tener mayor puntuacion
    return 1/value; // implemente aqui la funci�n que asigna valores a las celdas
}
\end{lstlisting}