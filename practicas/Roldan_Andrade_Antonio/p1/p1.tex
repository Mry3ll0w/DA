\documentclass[]{article}

\usepackage[left=2.00cm, right=2.00cm, top=2.00cm, bottom=2.00cm]{geometry}
\usepackage[spanish,es-noshorthands]{babel}
\usepackage[utf8]{inputenc} % para tildes y ñ
\usepackage{graphicx} % para las figuras
\usepackage{xcolor}
\usepackage{listings} % para el código fuente en c++

\lstdefinestyle{customc}{
  belowcaptionskip=1\baselineskip,
  breaklines=true,
  frame=single,
  xleftmargin=\parindent,
  language=C++,
  showstringspaces=false,
  basicstyle=\footnotesize\ttfamily,
  keywordstyle=\bfseries\color{green!40!black},
  commentstyle=\itshape\color{gray!40!gray},
  identifierstyle=\color{black},
  stringstyle=\color{orange},
}
\lstset{style=customc}


%opening
\title{Práctica 1. Algoritmos devoradores}
\author{Antonio Roldán Andrade \\ % mantenga las dos barras al final de la línea y este comentario
antonio.roldanandrade@alum.uca.es \\ % mantenga las dos barras al final de la línea y este comentario
Teléfono: 611404497 \\ % mantenga las dos barras al final de la linea y este comentario
NIF: 49562495W \\ % mantenga las dos barras al final de la línea y este comentario
}


\begin{document}

\maketitle

%\begin{abstract}
%\end{abstract}

% Ejemplo de ecuación a trozos
%
%$f(i,j)=\left\{ 
%  \begin{array}{lcr}
%      i + j & si & i < j \\ % caso 1
%      i + 7 & si & i = 1 \\ % caso 2
%      2 & si & i \geq j     % caso 3
%  \end{array}
%\right.$

\begin{enumerate}
\item Describa a continuación la función diseñada para otorgar un determinado valor a cada una de las celdas del terreno de batalla para el caso del centro de extracción de minerales. 

Escriba aquí su respuesta al ejercicio 1.

valor = (def.damage*def.attacksPerSecond*def.health*def.range) / (def.dispersion);

\item Diseñe una función de factibilidad explicita y descríbala a continuación.

Escriba aquí su respuesta al ejercicio 2.

Una matriz reprensentada con un vector de vectores de tipo float, de la siguiente forma,
std::vector<std::vector<float>>& TSP, cada vector representará uno los valores y otro los costes de las defensas,
respectivamente.

Una lista donde están almacenadas las defensas, que nos permite recorrerlas con mayor facilidad.
Nos permite también las inserciones y borrados con mayor rapidez.

\item A partir de las funciones definidas en los ejercicios anteriores diseñe un algoritmo voraz que resuelva el problema para el caso del centro de extracción de minerales. Incluya a continuación el código fuente relevante. 

Escriba aquí su respuesta al ejercicio 3.

\item Comente las características que lo identifican como perteneciente al esquema de los algoritmos voraces. 

\begin{lstlisting}
std::list<Defense*> recupera_defensas(const std::vector<std::vector<int> >& tsp,
const std::list<defensa_valoracion>& def_val,const int& filas, const int& cols,std::list<Defense*>defenses){

    //Almacenara las defensas que se han usado
    std::list<Defense*>sol;

    int i = filas - 2;
    int j = cols - (*defenses.begin())->cost; //Tenemos que eliminar el coste de la primera defense (centro de extraccion), que ademas se añade en cualquier caso

    List<Defense*>::iterator it = defenses.end();

    it--;   //Iniciamos en la posicion anterior a la ultima defensa (seria fin - 1)
    
    //Se Recorrera inversamente ya que partimos del beneficio maximo que se encuentra en 
    //la ultima posicion de la matriz
    while(i > 0)
    {
        if(tsp[i][j] != tsp[i-1][j])
        {
            j = j - (*it)->cost;
            sol.push_back(*(it));
        }
        i--;
        it--;
    }
    if(tsp[0][j] != 0) // En caso de que la primera posicion de la fila 0 sea != de 0 significa que la existe una defensa que tiene coste 0 y entrara en la lista
        sol.push_back((*it));//Insertamos dicha defensa

    
return sol; //Devolvemos la lista de defensas
}
\end{lstlisting}

\item Describa a continuación la función diseñada para otorgar un determinado valor a cada una de las celdas del terreno de batalla para el caso del resto de defensas. Suponga que el valor otorgado a una celda no puede verse afectado por la colocación de una de estas defensas en el campo de batalla. Dicho de otra forma, no es posible modificar el valor otorgado a una celda una vez que se haya colocado una de estas defensas. Evidentemente, el valor de una celda sí que puede verse afectado por la ubicación del centro de extracción de minerales.

La defincion de esta funcion se halla en el ejercicio 1, ya que ambos usan la misma.
El codigo en cuestion: 
\begin{lstlisting}
    float cellValue(int row, int col, bool** freeCells, int nCellsWidth, int nCellsHeight
	, float mapWidth, float mapHeight, List<Object*> obstacles, List<Defense*> defenses) {
	//De forma similar al ejercicio inferior
    float cellWidth = mapWidth / nCellsWidth; //anchura de la celula
    float cellHeight = mapHeight / nCellsHeight;//altura de la celula
    /*Aplicamos el criterio ==> cuanto mas cerca de un obstaculo mejor 
    */
    //Usamos el tipo vector 3 para una comparativa (si esta vacia o no de forma mas sencilla)
    Vector3 t_posicion = cellCenterToPosition(row,col,cellWidth,cellHeight);
    float value = 0;
    for(auto i: obstacles){
        value+=i->position.subtract(t_posicion).length();
        //value+=_distance(t_position,i.position); 
        //me da fallos a pesar de que esto representa lo mismo que arriba
    }
    //el que tenga menor valor tendra mas obstaculos cerca ==> mayor puntuacion
    //por tanto lo invierto para tener mayor puntuacion
    return 1/value; // implemente aqui la funci�n que asigna valores a las celdas
}
\end{lstlisting}

\item A partir de las funciones definidas en los ejercicios anteriores diseñe un algoritmo voraz que resuelva el problema global. Este algoritmo puede estar formado por uno o dos algoritmos voraces independientes, ejecutados uno a continuación del otro. Incluya a continuación el código fuente relevante que no haya incluido ya como respuesta al ejercicio 3. 

\begin{lstlisting}
void DEF_LIB_EXPORTED placeDefenses(bool **freeCells, int nCellsWidth, int nCellsHeight, float mapWidth,
                                    float mapHeight, std::list<Object *> obstacles, std::list<Defense *> defenses)
{

    float cellWidth = mapWidth / nCellsWidth;
    float cellHeight = mapHeight / nCellsHeight;

    int maxAttemps = 1000;
    std::vector<celda_valoracion> celdas_valoradas;
    int i, j, k;
    float aux;
    celda_valoracion auxCell;

    /* -------------------------------------------------------------------------- */
    /*                 ASINGACION DEL CENTRO DE EXTRACCION (DEF 0)                */
    /* -------------------------------------------------------------------------- */

    // 1) Obtenemos que celdas estan libres tras la colocacion de los obstaculos
    for (int i = 0; i < nCellsHeight; i++)
    {

        for (int j = 0; j < nCellsWidth; j++)
        {
            if (freeCells[i][j] != false)
            {

                celdas_valoradas.push_back(celda_valoracion(i, j, cellValueExtractionCenter(i, j, freeCells, nCellsWidth, nCellsHeight, mapWidth, mapHeight, obstacles)));
            }
        }
    }

    // Ordenacion de las celdas valoradas aplicando los constructores de vector y lista

    std::list<celda_valoracion> aux_lista(celdas_valoradas.begin(), celdas_valoradas.end()); // Creo una lista con los elementos de celdas
    aux_lista.sort();                                                                        // Ordeno los elementos O(n) = n · log n
    celdas_valoradas = std::vector<celda_valoracion>(aux_lista.begin(), aux_lista.end());    // los copio a la lista

    bool placed = false;
    celda_valoracion solution;
    List<Defense *>::iterator currentDefense = defenses.begin();

    // Algoritmo devorador para centro de extraccion
    while (!placed && !celdas_valoradas.empty())
    {
        solution = celdas_valoradas.back();
        celdas_valoradas.pop_back();
        if (funcion_factibilidad(defenses, *(*currentDefense), obstacles, mapHeight,
                                 cellWidth, cellHeight, mapWidth, solution.row, solution.col, freeCells))
        {
            placed = true;
            freeCells[solution.row][solution.col] = false;
            (*currentDefense)->position = cellCenterToPosition(solution.row, solution.col, cellWidth, cellHeight);
        }
    }
    // Copia de los candidatos

    std::vector<celda_valoracion> celdas_libres_aux;

    // Insertamos las celdas aun libres en el vector
    for (i = 0; i < nCellsHeight; i++)
        for (j = 0; j < nCellsWidth; j++)
        {
            if (freeCells[i][j])
                celdas_libres_aux.push_back(celda_valoracion(i, j, defaultCellValue(i, j, freeCells, nCellsWidth, nCellsHeight, mapWidth, mapHeight, obstacles, defenses)));
        }

    // Ordenamos los elementos de forma similar a lo hecho con anterioridad
    std::list<celda_valoracion> aux_lista2(celdas_libres_aux.begin(), celdas_libres_aux.end()); // Creo una lista con los elementos de celdas
    aux_lista2.sort();                                                                          // Ordeno los elementos O(n) = n · log n
    // Copiamos el contenido a las celdas originales para la insercion en el mapa de las defensas
    celdas_valoradas = std::vector<celda_valoracion>(aux_lista2.begin(), aux_lista2.end());
    ;

    std::vector<celda_valoracion>::iterator it;

    currentDefense++; // pasamos al segundo elemento ya que el primero ya se ha colocado (centro de extraccion)

    while (currentDefense != defenses.end())
    {

        placed = false;
        it = celdas_libres_aux.end();
        while (!placed && !celdas_valoradas.empty())
        {

            solution = celdas_valoradas.back();
            celdas_valoradas.pop_back();
            if (funcion_factibilidad(defenses, *(*currentDefense), obstacles, mapHeight, cellWidth, cellHeight, mapWidth, solution.row, solution.col, freeCells))
            {
                placed = true;
                (*currentDefense)->position = cellCenterToPosition(solution.row, solution.col, cellWidth, cellHeight);
                it--;
                celdas_libres_aux.erase(it);
            }
        }
        celdas_valoradas = celdas_libres_aux;
        currentDefense++;
    }
}
\end{lstlisting}

\end{enumerate}

Todo el material incluido en esta memoria y en los ficheros asociados es de mi autoría o ha sido facilitado por los profesores de la asignatura. Haciendo entrega de este documento confirmo que he leído la normativa de la asignatura, incluido el punto que respecta al uso de material no original.

\end{document}
